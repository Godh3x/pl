\providecommand{\main}{..}
\documentclass[\main/ApuntesPL.tex]{subfiles}

% !TEX root = ../ApuntesPL.tex

\begin{document}
  \begin{tikzpicture}
  \node[sheet, label = {[align=center]Input\\(Programa)} ] (input) at (0,0) {};
  \node[unit, right = of input, align = center] (anle) {Analizador\\léxico};
  \node[rectangle, draw = black, right = of anle, align = center, rectangle split, rectangle split horizontal, rectangle split parts = 2] (cl1) {};
  \node[rectangle, draw = black, right = .2 of cl1, align = center, rectangle split, rectangle split horizontal, rectangle split parts = 2] (cl2) {};
  \node[rectangle, draw = black, right = .2 of cl2, align = center, rectangle split, rectangle split horizontal, rectangle split parts = 2] (cl3) {};
  \node[txt, right = .2 of cl3] (cl_extra) {...};
  \node[txt, below = .2 of cl2] (cl_data1) {Secuencia de \textbf{tokens}\\(Componentes léxicos)};
  \node[txt, above = of cl1] (cl_data1) {Clase léxica (código)};
  \node[txt, right = .8 of cl_data1] (cl_data2) {Lexema (opcional). Tiene sentido si la clase\\léxica no determina \textbf{unívocamente} la cadena.\\\\\underline{Clase}\\\textbf{multivaluada}
        $\equiv$ \textbf{si} lexema\\\textbf{univaluada} $\equiv$ \textbf{no} lexema};



  \draw[myarrow] (input.east) -- ++(0,0) -> (anle.west);
  \draw[myarrow] (anle.east) -- ++(0,0) -> (cl1.west);
  \draw[myarrow2] (cl1.north) -- ++(-.2,0) -> (cl_data1.south);
  \draw[myarrow2] (cl3.north) -- ++(.2,0) -> (cl_data2.west);
  \end{tikzpicture}

  \section{e.g. prototípico} \label{egp}
    \begin{tikzpicture}
      \node[rectangle, label = Input] (input) at (0,0) {(5 + 7) / 8 - 9};
      \node[unit, right = of input, align = center] (calc) {Calculador};
      \node[rectangle, label = Output, right = of calc] (output) {-8};

      \node[unit, below = .8 of calc, align = center] (anle) {Analizador\\léxico};
      \node[rectangle, draw = black, right = of anle, align = center] (cl1) {PAP};
      \node[rectangle, draw = black, right = .2 of cl1, align = center, rectangle split, rectangle split horizontal, rectangle split parts = 2] (cl2) {NUM\nodepart{second}"5"};
      \node[rectangle, draw = black, right = .2 of cl2, align = center] (cl3) {MAS};
      \node[txt, right = .2 of cl3] (cl_extra) {...};


      \draw[myarrow] (input.east) -- ++(0,0) -> (calc.west);
      \draw[myarrow] (calc.east) -- ++(0,0) -> (output.west);

      \draw[myarrow2] (calc.south) -- ++(0,0) -> (anle.north);
      \draw[myarrow] (anle.east) -- ++(0,0) -> (cl1.west);
    \end{tikzpicture}

    \FloatBarrier
    \bigskip
    \par
    Una expresión aritmética es...
    \begin{itemize}
      \item ...un \textbf{número entero}.
      \item ...una \textbf{expresión} seguida de un \textbf{operador} seguido de otra \textbf{expresión}.
      \item...\textbf{(} seguido de una \textbf{expresión} seguido de \textbf{)}.
    \end{itemize}

    \par
    Los operadores son: \textbf{+}, \textbf{-}, \textbf{*} y \textbf{/}.

  \section{Método de desarrollo}
    \subsection{Identificación de las clases léxicas}
      \par
      Normalmente:
      \begin{itemize}
        \item Los \textbf{signos} (puntos, operadores...)  conviene que sean clases \textbf{univaluadas}.
        \item \textbf{Variables} (NUM,LETRA...) son \textbf{multivaluadas}.
        \item \textbf{Palabras reservadas} (bool, int...) son \textbf{univaluadas}.
      \end{itemize}

      \newpage
      \par
      \textbf{Clases léxicas de \cref{egp}.}
      \begin{itemize}
        \item PAP (paréntesis apertura)
        \item PCIERRE (paréntesis cierre)
        \item NUM
        \item MAS
        \item MENOS
        \item POR
        \item DIV
      \end{itemize}

      \bigskip
      \par
      \textbf{e.g.}
      \begin{center}
        \begin{minipage}{.2 \textwidth}
          \underline{class} \underline{C} \{\\
          \hspace*{8mm}\underline{int} \underline{x};\\
          \hspace*{8mm}....
        \end{minipage}%
        \begin{minipage}{.6 \textwidth}
          \begin{itemize}
            \item class $\equiv$ univaluada
            \item C $\equiv$ id
            \item \{ $\equiv$ univaluada
            \item int $\equiv$ univaluada
            \item x $\equiv$ id
          \end{itemize}
        \end{minipage}
      \end{center}

    \subsection{Describir las clases léxicas}
      \par
      Cada clase léxica es un lenguaje formal de los posibles lexemas.\\
      \textbf{Suposición}: son lenguajes regulares, por tanto:
      \begin{itemize}
        \item Se pueden describir mediante $ER_s$ (expresiones regulares).
        \item Se pueden reconocer utilizando $AFD_s$ (autómatas finitos deterministas).
      \end{itemize}

      \bigskip
      \par
      \textbf{e.g.} NENT
      \begin{itemize}
        \item Descripción informal.
          \par
          Empieza con un signo (+ o -) opcional.\\
          A continuación aparecen uno o más dígitos.\\
          e.g. +5, +007, 000, -0, -08, -280.
        \item Descripción formal.
          \par
          (\textbackslash+ $\mid$ \textbackslash -)?[0-9]+
      \end{itemize}

      \newpage
      \par
      \textbf{\large Notación}
      \begin{itemize}
        \item ER sobre el alfabeto $\Sigma$ (e.g. UNICODE, ASCII...).
              \begin{itemize}
                \item Una "letra" de $\Sigma$, e.g. a. Denota \{a\}
                \item Si $E_0$ y $E_1$ son $ER_s$, también lo son:
                  \begin{itemize}
                    \item $(E_0 \mid E_1)$. Denota la unión, $L(E_0) \cup L(E_1).$
                    \item $(E_0 \bullet E_1)$. Denota la concatenación, $ \{W_0 W_1 \mid W_0 \in  L(E_0), W_1 \in L(E_1)\}$.
                    \item $(E_0 \ast)$. Denota \{$\epsilon, W_0, W_0 W_1, W_0 W_1 W_2 ...\mid W_0, W_1, W_2, ... \in L(E_0)\}$.
                  \end{itemize}
                \item $\epsilon$ denota el lenguaje formado por la cadena vacía.
              \end{itemize}
        \item Convenios.
              \begin{itemize}
                \item $\ast$ tiene mayor prioridad que $\mid$ y $\bullet$.
                \item $\bullet$ tiene mayor prioridad que $\mid$.
                \item $\bullet$ se puede omitir.
                \item Los ( ) se usan para cambiar prioridades.
                \item Conjuntos de caracteres, e.g. [0-9,a] es el conjunto formado por los dígitos y la a.
                  \begin{itemize}
                    \item $[e_0, ..., e_n]$ donde $e_i$ puede ser...
                      \begin{itemize}
                        \item ...una letra.
                        \item ...a-b, conjunto de caracteres comprendidos entre a y b.
                      \end{itemize}
                    \item Conjuntos complementados [$^\wedge e_0, ..., e_n$].
                        \par
                        e.g. [$^\wedge$0-9, a-z] es el conjunto de caracteres que no son dígitos ni letras minúsculas.
                    \item E+ $\equiv$ EE$\ast$. Aparece una o más veces pero mínimo una.
                    \item E? $\equiv$ (E$\mid\epsilon$). Aparece o no, es opcional.
                    \item \textbackslash. La forma de escape.
                  \end{itemize}
              \end{itemize}
      \end{itemize}

      \bigskip
      \par
      \textbf{e.g.} Identificadores
      \begin{itemize}
      \item Pueden contener letras, dígitos y \_.
      \item Empiezan por letra o \_.
      \item Van seguidos de una secuencia de 0 o más caracteres válidos.
      \end{itemize}
      \hspace{5mm}[a-z, A-Z, \_] [a-z, A-Z, \_, 0-9]$\ast$

      \bigskip
      \par
      \textbf{e.g.} NENT, como en el ejemplo anterior pero sin $0_s$ a la izquierda.\\
      \hspace{5mm} [\textbackslash+, \textbackslash -]?([1-9] [0-9]$\ast \mid$ 0)

      \newpage
      \par
      Para mayor claridad vamos a utilizar $DR_s$ (definiciones regulares) en lugar de $ER_s$.
      \begin{itemize}
        \item ($\ast$) $\equiv$ clase léxica, marca cual es la definición principal.
        \item $[I]$ $\equiv$ ignorables, se utiliza para definir que caracteres no se han de tener en cuenta.
        \item Las palabras subrayadas corresponden a definiciones auxiliares.
      \end{itemize}

      \bigskip
      \par
      \textbf{e.g.} Identificadores\\
      \hspace{5mm}($\ast$) IDEN $\equiv$ \underline{Letra} (\underline{Letra} $\mid$ \underline{Dig})$\ast$ \\
      \hspace{10mm} Letra $\equiv$ [a-z, A-Z, \_] \\
      \hspace{10mm} Dig $\equiv$ [0-9]

      \bigskip
      \par
      \textbf{e.g.} Números enteros\\
      \hspace{5mm}($\ast$) LENT $\equiv$ \underline{Signo}? (0 $\mid$ \underline{DPos} \underline{Dig}$\ast$) \\
      \hspace{10mm} Signo $\equiv$ [\textbackslash+, \textbackslash-] \\
      \hspace{10mm} DPos  $\equiv$ [1-9] \\
      \hspace{10mm} Dig   $\equiv$ [0-9]

      \bigskip
      \par
      \textbf{e.g.} Literales reales
      \begin{itemize}
        \item Empiezan por un entero.
        \item Continúan por...
          \begin{itemize}
            \item ...una parte decimal.
            \item ...una parte exponencial.
            \item ...una parte decimal seguida de una exponencial.
          \end{itemize}
        \item Parte decimal. '.' seguido de una secuencia de uno o más dígitos, sin ceros superfluos a la derecha.
        \item Parte exponencial. E o e seguida de un entero.
      \end{itemize}
      \hspace{10mm}e.g. +5.7E-28\\
      \bigskip
      \hspace{5mm}($\ast$) LREAL $\equiv$ \underline{LENT} (\underline{PDEC} $\mid$ \underline{PEXP} $\mid$ \underline{PDEC} \underline{PEXP}) \\
      \hspace{10mm} PDEC $\equiv$ \textbackslash. (\underline{Dig}$\ast$ \underline{DPos} $\mid$ 0 ) \\
      \hspace{10mm} PEXP $\equiv$ (E $\mid$ e) \underline{LENT} \\
      \vspace{2mm}
      \hspace{10mm} LENT $\equiv$ \underline{Signo}? (0 $\mid$ \underline{DPos} \underline{Dig}$\ast$) \\
      \hspace{10mm} Signo $\equiv$ [\textbackslash+, \textbackslash-] \\
      \hspace{10mm} DPos  $\equiv$ [1-9] \\
      \hspace{10mm} Dig   $\equiv$ [0-9]

      \newpage
      \par
      \textbf{\large Ignorables} \\
      \hspace{3mm}$[I]$ SEP $\equiv$ $[$' ', \textbackslash t, \textbackslash n, \textbackslash r, \textbackslash b], posibles separadores. \\
      \hspace{3mm}$[I]$ COM $\equiv$ \# [$^\wedge$ \textbackslash n]$\ast$ \textbackslash n, e.g. \# Esto es un comentario.\textbackslash n

      \bigskip
      \par
      \textbf{e.g}  DR de \cref{egp}\\
      \hspace{5mm}($\ast$) PAP $\equiv$ \textbackslash( \\
      \hspace{5mm}($\ast$) PCIERRE $\equiv$ \textbackslash) \\
      \hspace{5mm}($\ast$) NUM $\equiv$ \underline{Signo}? (0 $\mid$ \underline{DPos} \underline{Dig}$\ast$) \\
      \hspace{10mm} Signo $\equiv$ [\textbackslash+, \textbackslash-] \\
      \hspace{10mm} DPos  $\equiv$ [1-9] \\
      \hspace{10mm} Dig   $\equiv$ [0-9] \\
      \hspace{5mm}($\ast$) MAS $\equiv$ \textbackslash + \\
      \hspace{5mm}($\ast$) MENOS $\equiv$ \textbackslash - \\
      \hspace{5mm}($\ast$) POR $\equiv$ \textbackslash $\ast$ \\
      \hspace{5mm}($\ast$) DIV $\equiv$ \textbackslash / \\
      \hspace{5mm}$[I]$ SEP $\equiv$ $[$' ', \textbackslash t, \textbackslash n, \textbackslash r, \textbackslash b]\\
      \hspace{5mm}$[I]$ COM $\equiv$ \# [$^\wedge$ \textbackslash n]$\ast$ \textbackslash n

  \section{Implementación}
    \subsection{Generadores de analizadores léxicos}
      \begin{tikzpicture}
      \node[sheet, label = Input] (input) at (0,0) {};
      \node[txt, below = .2 of input] (input_extra) {Especificación del\\analizador léxico\\ basado en $DR_s$};
      \node[unit, right = of input, align = center] (gen) {Generador};
      \node[txt, right = of input_extra] (gen_extra) {ER};
      \node[txt, right = 2 of gen_extra] (gen_extra2) {AFN};
      \node[txt, right = 2 of gen_extra2] (gen_extra3) {AFD};
      \node[txt, right = 2 of gen_extra3] (gen_extra4) {$AFD_{minimo}$};
      \node[clo, label = Output, right = of gen] (output) {};
      \node[txt, right = .2 of output] (output_extra) {Analizador léxico escrito en\\lenguaje de alto nivel, e.g. Java.};

      \draw[myarrow] (input.east) -- ++(0,0) -> (gen.west);
      \draw[myarrow] (gen.east) -- ++(0,0) -> (output.west);
      \draw[myarrow2] (gen.south) -- ++(-.4,0) -> (gen_extra.west);

      \draw[myarrow, align = center] (gen_extra.east) -- ++(0,0) -> (gen_extra2.west)  node[midway, below] {construcción\\de Thompson};
      \draw[myarrow, align = center] (gen_extra2.east) -- ++(0,0) -> (gen_extra3.west)  node[midway, below] {construcción\\de subconjuntos};
      \draw[myarrow] (gen_extra3.east) -- ++(0,0) -> (gen_extra4.west) node[midway, below] {minimización};
      \end{tikzpicture}

    \subsection{Implementación manual}
      \par
      Se implementa utilizando $AFD_s$ donde las transiciones a estados de error no se dibujan,
      son implicitas.

      \bigskip
      \par
      El AFD no reconoce la entrada completamente sino que utiliza una \textbf{arquitectura pull}, solicita
      los token uno a uno y los va procesando.

      \vspace{5mm}
      \begin{tikzpicture}
        \node[txt] (cl_extra) at (0,0) {...};
        \node[rectangle, draw = black, right = .2 of cl_extra, align = center, rectangle split, rectangle split horizontal, rectangle split parts = 2] (cl1) {};
        \node[rectangle, draw = black, right = .2 of cl1, align = center, rectangle split, rectangle split horizontal, rectangle split parts = 2] (cl2) {};
        \node[rectangle, draw = black, right = .2 of cl2, align = center, rectangle split, rectangle split horizontal, rectangle split parts = 2] (cl3) {};

        \node[unit, right = of cl3, align = center] (anle) {Analizador\\léxico};


        \node[rectangle, draw = black, right = of anle, align = center, rectangle split, rectangle split horizontal, rectangle split parts = 2] (cl4) {};
        \node[rectangle, draw = black, right = .2 of cl4, align = center, rectangle split, rectangle split horizontal, rectangle split parts = 2] (cl5) {};
        \node[rectangle, draw = black, right = .2 of cl5, align = center, rectangle split, rectangle split horizontal, rectangle split parts = 2] (cl6) {};
        \node[txt, right = .2 of cl6] (cl_extra) {...};

        \draw[myarrow] (cl3.east) -- ++(0,0) -> (anle.west);
        \draw[myarrow] (anle.east) -- ++(0,0) -> (cl4.west);
      \end{tikzpicture}

      \bigskip
      \par
      Siempre hay que tener una clase léxica que represente el EOF.

      \newpage
      \par
      \textbf{e.g.} AFD de \cref{egp}

      \begin{tikzpicture}[shorten >=1pt,node distance=2cm,on grid,auto]
        \node[state] (q_0)   {Inicio};
        \node[state,accepting](q_7) [left=of q_0] {\scriptsize REC0};
        \node[state,accepting](q_3) [above=of q_7] {\scriptsize RECMAS};
        \node[state,accepting](q_4) [below=2.5of q_7] {\scriptsize RECMENOS};
        \node[state,accepting](q_8) [left=3of q_7] {\scriptsize RECNUM};
        \node[state](q_10) [right=4of q_3] {\scriptsize RECEOF};
        \node[state](q_9) [below=4of q_0] {\scriptsize RECCOM};
        \node[state, accepting] (q_1) [right=4of q_4] {\scriptsize RECPAP};
        \node[state, accepting] (q_2) [right=of q_1] {\scriptsize RECPC};
        \node[state,accepting](q_6) [above right=of q_2] {\scriptsize RECDIV};
        \node[state,accepting](q_5) [below right=of q_10] {\scriptsize RECPOR};

        \draw[<-] (q_0) -- node[below] {start} ++(0,-1cm);
        \path[->]
          (q_0) edge [loop above] node[rotate=-20]         {\scriptsize$[$' ',\textbackslash t,\textbackslash n,\textbackslash r,\textbackslash b]} (q_0)
                edge              node[above]              {\scriptsize +} (q_3)
                edge              node[near end, above]    {\scriptsize -} (q_4)
                edge              node[above]              {\scriptsize 0} (q_7)
                edge  [bend left] node[near end, above]    {\scriptsize [1-9]} (q_8)
                edge              node[below, rotate = 45] {\scriptsize EOF} (q_10)
                edge [bend right] node[near end]           {\scriptsize \#} (q_9)
                edge              node[near end]           {\scriptsize (} (q_1)
                edge              node[near end]           {\scriptsize )} (q_2)
                edge              node[near end]           {\scriptsize /} (q_6)
                edge              node[near end]           {\scriptsize $\ast$} (q_5)
          (q_4) edge              node[near start]         {\scriptsize 0} (q_7)
                edge              node                     {\scriptsize [1-9]} (q_8)
          (q_3) edge              node                     {\scriptsize 0} (q_7)
                edge              node                     {\scriptsize [1-9]} (q_8)
          (q_9) edge [bend right] node[near start]         {\scriptsize \textbackslash n} (q_0)
                edge [loop right] node[right]              {[$^\wedge$ \textbackslash n]} (q_9);
      \end{tikzpicture}

      \FloatBarrier
      \hspace{5mm}
      \par
      \textbf{Inicialización:} Conseguir el primer caracter en sigCar.

      \begin{center}
        \begin{minipage}{.35 \textwidth}
          \par
          \textbf{SigToken}\\
          \hspace*{5mm}Estado $\leftarrow$ EstadoInicial\\
          \hspace*{5mm}lexema $\leftarrow$ ""\\
          \hspace*{5mm}\textbf{loop} \{\\
          \hspace*{10mm}\textbf{switch}(Estado) \{\\
          \hspace*{15mm}\textbf{case} $S_0$\\
          \hspace*{15mm}...\\
          \hspace*{15mm}\textbf{case} $S_n$\\
          \hspace*{10mm}\}\\
          \hspace*{5mm}\}
        \end{minipage}%
        \begin{minipage}{.65 \textwidth}
          \bigskip
          \par
          \textbf{case $S_i$}\\
          \hspace*{5mm}\textbf{si} sigCar $\in$  $T_0$\\
          \hspace*{10mm}Estado $\leftarrow$ SigEstado\\
          \hspace*{10mm}lexema $\leftarrow$ lexema + sigCar\\
          \hspace*{15mm}(lexema solo cambia en estados no ignorables)\\
          \hspace*{10mm}Actualizar sigCar\\
          \hspace*{5mm}\textbf{si no si} sigCar $\in$  $T_1$\\
          \hspace*{10mm}Estado $\leftarrow$ SigEstado\\
          \hspace*{10mm}lexema $\leftarrow$ lexema + sigCar\\
          \hspace*{10mm}Actualizar sigCar\\
          \hspace*{5mm}...\\
          \hspace*{5mm}\textbf{si no}\\
          \hspace*{10mm}//Si es un estado final\\
          \hspace*{10mm}\textbf{return} token\\
          \hspace*{10mm}//Si no es un estado final\\
          \hspace*{10mm}\textbf{error}
        \end{minipage}
      \end{center}

      \bigskip
      \par
      Dos variables extra, fila y columna, para poder dar información extra en el mensaje de error.

      \newpage
      \par
      \textbf{Codificación parcial (pseudocódigo)}\\
      \vspace{5mm}
      \hspace{5mm}Estado $\leftarrow$ Inicio\\
      \hspace{5mm}lexema $\leftarrow$ ""\\
      \hspace{5mm}\textbf{loop} \{\\
      \hspace{10mm}\textbf{switch} (Estado) \{\\
      \hspace{10mm}Inicio: \textbf{if} sigCar = ( \textbf{then} transita(RECPAP)\\
      \hspace{22mm}\textbf{else if} ...\\
      \hspace{22mm}...\\
      \hspace{22mm}\textbf{else if} sigCar = \# textbf{then} transitaIgnorando(RECCOM)\\
      \hspace{22mm}\textbf{else if} sigCar $\in$ [1-9] \textbf{then} transita(RECNUM)\\
      \hspace{22mm}\textbf{else} error()\\
      \vspace{2mm}
      \hspace{10mm}RECNUM: \textbf{if} sigCar $\in$ [1-9] \textbf{then} transita(RECNUM)\\
      \hspace{30mm}\textbf{else return} token(NUM, lexema)\\
      \vspace{2mm}
      \hspace{10mm}...\\
      \vspace{2mm}
      \hspace{10mm}RECCOM: \textbf{if} sigCar $\neq$ \textbackslash n \textbf{then} transitaIgnorando(RECCOM)\\
      \hspace{30mm}\textbf{else} transitaIgnorando(Inicio)\\
      \hspace{10mm}\}\\
      \hspace{5mm}\}\\
      \vspace{10mm}
      \hspace{5mm}transita(S) \{\\
      \hspace{10mm}Estado $\leftarrow$ S\\
      \hspace{10mm}lexema $\leftarrow$ lexema + sigCar\\
      \hspace{10mm}Actualiza sigCar\\
      \hspace{5mm}\}\\
      \vspace{10mm}
      \hspace{5mm}transitaIgnorando(S) \{\\
      \hspace{10mm}Estado $\leftarrow$ S\\
      \hspace{10mm}Actualiza sigCar\\
      \hspace{5mm}\}

      \newpage
      \par
      \textbf{e.g.}
      \begin{center}
        \begin{minipage}{.2 \textwidth}
          EVALUA\\
          5 + x\\
          DONDE\\
          x = 27
        \end{minipage}%
        \begin{minipage}{.5 \textwidth}
          \par
          \textbf{Especificación léxica}\\
          \vspace{2mm}
          \hspace{5mm}...\\
          \hspace{5mm}($\ast$) ID $\equiv$ \underline{Letra} (\underline{Letra} $\mid$ \underline{Dig})$\ast$\\
          \hspace{11mm}Letra $\equiv$ $[$a-z,A-Z,\_]\\
          \hspace{11mm}Dig $\equiv$ $[$0-9]\\
          \hspace{5mm}($\ast$) EVALUA $\equiv$ [E,e][V,v][A,a][L,l][U,u][A,a]\\
          \hspace{5mm}($\ast$) DONDE $\equiv$ [D,d][O,o][N,n][D,d][E,e]\\
        \end{minipage}
      \end{center}

      \bigskip
      \par
      En el return que reconoce los id se comprueba si el lexema es una palabra reservada, de serlo se devuelve la palabra reservada en lugar del id.\\
      \vspace{5mm}
      \begin{tikzpicture}[shorten >=1pt,node distance=2cm,on grid,auto]
        \node[state,initial] (q_0)   {Inicio};
        \node[state,accepting](q_1) [right=3of q_0] {\scriptsize ID};

        \path[->]
          (q_0) edge              node        {Letra$\mid$Dig} (q_1)
          (q_1) edge [loop right] node[right] {Letra$\mid$Dig} (q_1);
      \end{tikzpicture}

      \bigskip
      \par
      Hay un ejemplo de codificación en el campus, tiene una errata en el diagrama de transiciones.

    \subsection{Implementación mediante herramientas}
      \par
      La herramienta que vamos a utilizar es \textbf{JLex}, hay un ejemplo en el CV.

      \bigskip
      \par
      Formato de JLex:
      \FloatBarrier
      \vspace{3mm}
      \begin{tikzpicture}
        \node[unit, align = center, minimum width = 5.1cm] (cfg) {Configuracion\\Puede incluir código en Java};
        \node[unit, below=.2of cfg, align = center, minimum width = 5.1cm] (drs) {Definiciones regulares};
        \node[unit, below=.2of drs, align = center, minimum width = 5.1cm] (act) {Patrones - acciones};
      \end{tikzpicture}

      \bigskip
      \par
      Para que una DR pueda referirse a otras tienen que haberse definido antes, lo que en las $DR_s$ subrayamos tiene que ir entre \{\}, hacen sustituciones literales (no ponen paréntesis).

      \bigskip
      \par
      Las palabras reservadas se ponen sin mas, el orden de reconocimiento equivale al de aparición en el fichero, en la parte de los return.

      \bigskip
      \par
      \textbf{Compilar:} java -cp jlex.jar JLex.Main input($DR_s$)
\end{document}