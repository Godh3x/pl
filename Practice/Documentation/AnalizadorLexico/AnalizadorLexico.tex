\providecommand{\main}{..}
\documentclass[\main/MemoriaPL.tex]{subfiles}

% !TEX root = ../MemoriaPL.tex

\begin{document}
  \section{Clases léxicas} \label{lexicon}
    \par
    La descripción de las clases léxicas identificadas se hará de manera informal, en lenguaje natural.

    \begin{itemize}
      \item \textbf{SPROG}
        \par
        Es el separador \&\& que indica el fin de la sección de declaracionesy el comienzo de la de instrucciones.
      \item \textbf{LREAL}
        \par
        Empiezan con un signo (+ o -) opcional, a continuación aparecen uno o más dígitos cualesquiera.
        Seguida de esta parte puede aparecer una decimal que consta de un punto seguido de uno o más dígitos cualesquiera.
        Por último, tiene una E o e seguida de un signo (+ o -), opcional, y de uno o más dígitos cualesquiera.
      \item \textbf{ID}
        \par
        Comienza por una letra cualquiera y la sigue una secuencia de cero o más letras, dígitos o subrayado(\_).
      \item \textbf{BOOL}
        \par
        Es una palabra reservada que se conforma por las letras minúsculas: b, o, o, l. En ese orden.
      \item \textbf{NUM}
        \par
        Palabra reservada formada por las letras minúsculas: n, u, m. En ese orden.
      \item \textbf{TRUE}
        \par
        Es una palabra reservada compuesta por las letras minúsculas: t, r, u, e. En ese orden.
      \item \textbf{FALSE}
        \par
        Palabra reservada que contiene las siguientes letras minúsculas: f, a, l, s, e. En ese orden.
      \item \textbf{PLUS}
        \par
        Representa ua suma, \textbackslash +.
      \item \textbf{MINUS}
        \par
        Representa una resta, \textbackslash -.
      \item \textbf{MUL}
        \par
        Representa la multiplicación, \textbackslash $\ast$.
      \item \textbf{DIV}
        \par
        Representa la división, /.
      \item \textbf{IS}
        \par
        Representación de la asignación, =.
      \item \textbf{EQ}
        \par
        Representa una comparación, ==.
      \item \textbf{GT}
        \par
        Representa el mayor que, $>$.
      \item \textbf{GEQ}
        \par
        Representa el mayor o igual que, $>$=.
      \item \textbf{LT}
        \par
        Representa el menor que, $<$.
      \item \textbf{LEQ}
        \par
        Representa el menor o igual que, $<$=.
      \item \textbf{NEQ}
        \par
        Representa una desigualdad, !=.
      \item \textbf{AND}
        \par
        Representa el operador lógico and.
      \item \textbf{OR}
        \par
        Representa el operador lógico or.
      \item \textbf{NOT}
        \par
        Representa el operador lógico not.
      \item \textbf{POP}
        \par
        Representa un paréntesis de apertura, $($.
      \item \textbf{PCL}
        \par
        Representa un paréntesis de cierre, $)$.
    \end{itemize}

  \section{Especificación formal}
    \par
    Vamos a utilizar $DR_s$ para dar una descrición formal del lenguaje que conforman las clases léxicas del \cref{lexicon}.

    \vspace{2mm}
    \hspace{5mm}($\ast$) SPROG $\equiv$ \&\& \\
    \vspace{2mm}
    \hspace{5mm}($\ast$) LREAL $\equiv$ \underline{LENT} \underline{PDEC}? \underline{PEXP}? \\
      \hspace{10mm} PDEC $\equiv$ \textbackslash. \underline{Dig}$\ast$ \underline{Dig} \\
      \hspace{10mm} PEXP $\equiv$ (E $\mid$ e) \underline{LENT} \\
      \vspace{2mm}
      \hspace{10mm} LENT $\equiv$ \underline{Signo}? \underline{Dig}$\ast$ \underline{Dig} \\
      \hspace{10mm} Signo $\equiv$ [\textbackslash+, \textbackslash-] \\
      \hspace{10mm} Dig   $\equiv$ [0-9] \\
      \vspace{2mm}
    \hspace{5mm}($\ast$) ID $\equiv$ \underline{Letra} (\underline{Letra} $\mid$ \underline{Dig} $\mid$ \_)$\ast$ \\
      \hspace{10mm} Letra $\equiv$ [a-z, A-Z] \\
      \hspace{10mm} Dig $\equiv$ [0-9] \\
    \vspace{2mm}
      \hspace{5mm}($\ast$) BOOL $\equiv$ b o o l \\
    \vspace{2mm}
    \hspace{5mm}($\ast$) NUM $\equiv$ n u m \\
    \vspace{2mm}
    \hspace{5mm}($\ast$) TRUE $\equiv$ t r u e \\
    \vspace{2mm}
    \hspace{5mm}($\ast$) FALSE $\equiv$ f a l s e \\
    \vspace{2mm}
    \hspace{5mm}($\ast$) PLUS $\equiv$ \textbackslash + \\
    \vspace{2mm}
    \hspace{5mm}($\ast$) MINUS $\equiv$ \textbackslash - \\
    \vspace{2mm}
    \hspace{5mm}($\ast$) MUL $\equiv$ \textbackslash $\ast$ \\
    \vspace{2mm}
    \hspace{5mm}($\ast$) DIV $\equiv$ / \\
    \vspace{2mm}
    \hspace{5mm}($\ast$) IS $\equiv$ = \\
    \vspace{2mm}
    \hspace{5mm}($\ast$) EQ $\equiv$ == \\
    \vspace{2mm}
    \hspace{5mm}($\ast$) GT $\equiv$ $>$ \\
    \vspace{2mm}
    \hspace{5mm}($\ast$) GEQ $\equiv$ $>$= \\
    \vspace{2mm}
    \hspace{5mm}($\ast$) LT $\equiv$ $<$ \\
    \vspace{2mm}
    \hspace{5mm}($\ast$) LEQ $\equiv$ $<$= \\
    \vspace{2mm}
    \hspace{5mm}($\ast$) NEQ $\equiv$ != \\
    \vspace{2mm}
    \hspace{5mm}($\ast$) AND $\equiv$ a n d \\
    \vspace{2mm}
    \hspace{5mm}($\ast$) OR $\equiv$ o r \\
    \vspace{2mm}
    \hspace{5mm}($\ast$) NOT $\equiv$ n o t \\
    \vspace{2mm}
    \hspace{5mm}($\ast$) POP $\equiv$ \textbackslash $($ \\
    \vspace{2mm}
    \hspace{5mm}($\ast$) PCL $\equiv$ \textbackslash $)$

  \section{Diagrama de transiciones} %TODO


\end{document}