\providecommand{\main}{..}
\documentclass[\main/MemoriaPL.tex]{subfiles}

% !TEX root = ../MemoriaPL.tex

\begin{document}
  \section{Clases léxicas}
    \par \noindent
    La descripción de las clases léxicas identificadas se hará de manera informal, en lenguaje natural.

    \begin{itemize}
      \item \textbf{PROG}
            \par \noindent
            El programa esta formado por una sección de declaraciones seguida del simbolo \&\&, tras este símbolo aparece
            la sección de instrucciones.
      \item \textbf{SDEC}
            \par \noindent
            Sección de declaraciones, la conforman una o más declaraciones separadas por ;.
      \item \textbf{SINS}
            \par \noindent
            Sección de instrucciones formada por una o más instrucciones separadas por ;.
      \item \textbf{DEC}
            \par \noindent
            Comienza por una palabra reservada que representa un tipo y la sigue un identificador.
      \item \textbf{INS}
            \par \noindent
            Comienza por un identificador de variable seguido del simbolo =, tras este aparecerá una expresión.
      \item \textbf{EXP}
            \par \noindent
            Una expresión básica consiste en un numero, true o false. También puede entenderse una expresión como 
            una expresión seguida de un operador aritmético o lógico seguido de otra expresión.
      \item \textbf{NUM}
            \par \noindent
            Empiezan con un signo (+ o -) opcional, a continuación aparecen uno o más dígitos cualesquiera.
            Seguida de esta parte puede aparecer una decimal que consta de un punto seguido de uno o más dígitos cualesquiera.
            Por último, tiene una E o e seguida de un signo (+ o -), opcional, y de uno o más dígitos cualesquiera.
      \item \textbf{ID}
            \par \noindent
            Comienza por una letra cualquiera y la sigue una secuencia de cero o más letras, dígitos o subrayado(\_).
      \item \textbf{BOOL}
            \par \noindent
            Es una palabra reservada que se conforma por las letras minúsculas: b, o, o, l. En ese orden.
      \item \textbf{NUM}
            \par \noindent
            Palabra reservada formada por las letras minúsculas: n, u, m. En ese orden.
      \item \textbf{TRUE}
            \par \noindent
            Es una palabra reservada compuesta por las letras minúsculas: t, r, u, e. En ese orden.
      \item \textbf{FALSE}
            \par \noindent
            Palabra reservada que contiene las siguientes letras minúsculas: f, a, l, s, e. En ese orden.
      \item \textbf{PLUS}
            \par \noindent
            Representa ua suma, \textbackslash +.
      \item \textbf{MINUS}
            \par \noindent
            Representa una resta, \textbackslash -.
      \item \textbf{MUL}
            \par \noindent
            Representa la multiplicación, \textbackslash $\ast$.
      \item \textbf{DIV}
            \par \noindent
            Representa la división, /.
      \item \textbf{IS}
            \par \noindent
            Representación de la asignación, =.
      \item \textbf{EQ}
            \par \noindent
            Representa una comparación, ==.
      \item \textbf{GT}
            \par \noindent
            Representa el mayor que, $>$.
      \item \textbf{GEQ}
            \par \noindent
            Representa el mayor o igual que, $>$=.
      \item \textbf{LT}
            \par \noindent
            Representa el menor que, $<$.
      \item \textbf{LEQ}
            \par \noindent
            Representa el menor o igual que, $<$=.
      \item \textbf{NEQ}
            \par \noindent
            Representa una desigualdad, !=.
      \item \textbf{POP}
            \par \noindent
            Representa un paréntesis de apertura, $($.
      \item \textbf{PCL}
            \par \noindent
            Representa un paréntesis de cierre, $)$.
    \end{itemize}

\section{Especificación formal} %TODO


\section{Diagrama de transiciones} %TODO


\end{document}