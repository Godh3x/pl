\providecommand{\main}{..}
\documentclass[\main/MemoriaPL.tex]{subfiles}

% !TEX root = ../MemoriaPL.tex

\begin{document}
  \section{Clases léxicas}
    \par \noindent
    La descripción de las clases léxicas identificadas se hará de manera informal, en lenguaje natural.

    \begin{itemize}
      \item \textbf{NUM}
            \par \noindent
            Empiezan con un signo (+ o -) opcional, a continuación aparecen uno o más dígitos cualesquiera.
            Seguida de esta parte puede aparecer una decimal que consta de un punto seguido de uno o más dígitos cualesquiera.
            Por último, tiene una E o e seguida de un signo (+ o -), opcional, y de uno o más dígitos cualesquiera.
      \item \textbf{ID}
            \par \noindent
            Comienza por una letra cualquiera y la sigue una secuencia de cero o más letras, dígitos o subrayado(\_).
      \item \textbf{BOOL}
            \par \noindent
            Es una palabra reservada que se conforma por las letras minúsculas: b, o, o, l. En ese orden.
      \item \textbf{NUM}
            \par \noindent
            Palabra reservada formada por las letras minúsculas: n, u, m. En ese orden.
      \item \textbf{TRUE}
            \par \noindent
            Es una palabra reservada compuesta por las letras minúsculas: t, r, u, e. En ese orden.
      \item \textbf{FALSE}
            \par \noindent
            Palabra reservada que contiene las siguientes letras minúsculas: f, a, l, s, e. En ese orden.
      \item \textbf{PLUS}
            \par \noindent
            Representa el signo \textbackslash +.
      \item \textbf{MINUS}
            \par \noindent
            Representa el signo \textbackslash -.
      \item \textbf{MUL}
            \par \noindent
            Representa el signo \textbackslash $\ast$.
      \item \textbf{DIV}
            \par \noindent
            Representa el signo /.
    \end{itemize}

\section{Especificación formal} %TODO


\section{Diagrama de transiciones} %TODO


\end{document}