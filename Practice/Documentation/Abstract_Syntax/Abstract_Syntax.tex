\providecommand{\main}{..}
\documentclass[\main/MemoriaPL.tex]{subfiles}

% !TEX root = ../MemoriaPL.tex

\begin{document}
  \section{Funciones constructoras}
    \par
    El primer paso para poder generar las funciones constructoras es simplificar
    la gramática incontextual generada en la fase 2.

    \bigskip
    \par
    Prog $\rightarrow$ LDec \underline{SPROG} LIns\\
    LDec $\rightarrow$ LDec \underline{EOL} \underline{NUM} \underline{ID}\\
    \hspace{13mm}$\mid$ LDec \underline{EOL} \underline{BOOL} \underline{ID}\\
    \hspace{13mm}$\mid$ \underline{NUM} \underline{ID}\\
    \hspace{13mm}$\mid$ \underline{BOOL} \underline{ID}\\
    LIns $\rightarrow$ LIns \underline{EOL} \underline{ID} \underline{IS} Exp\\
    \hspace{12mm}$\mid$ \underline{ID} \underline{IS} Exp\\
    Exp $\rightarrow$ Exp \underline{PLUS} Exp\\
    \hspace{11mm}$\mid$ Exp \underline{MINUS} Exp\\
    \hspace{11mm}$\mid$ Exp \underline{AND} Exp\\
    \hspace{11mm}$\mid$ Exp \underline{OR} Exp\\
    \hspace{11mm}$\mid$ Exp \underline{EQ} Exp\\
    \hspace{11mm}$\mid$ Exp \underline{GT} Exp\\
    \hspace{11mm}$\mid$ Exp \underline{GEQ} Exp\\
    \hspace{11mm}$\mid$ Exp \underline{LT} Exp\\
    \hspace{11mm}$\mid$ Exp \underline{LEQ} Exp\\
    \hspace{11mm}$\mid$ Exp \underline{NEQ} Exp\\
    \hspace{11mm}$\mid$ Exp \underline{MUL} Exp\\
    \hspace{11mm}$\mid$ Exp \underline{DIV} Exp\\
    \hspace{11mm}$\mid$ \underline{MINUS} Exp\\
    \hspace{11mm}$\mid$ \underline{NOT} Exp\\
    \hspace{11mm}$\mid$ \underline{LREAL}\\
    \hspace{11mm}$\mid$ \underline{TRUE}\\
    \hspace{11mm}$\mid$ \underline{FALSE}\\
    \hspace{11mm}$\mid$ \underline{POP} Exp \underline{PCL}\\

    \bigskip
    \par
    Por cada una de las reglas de esta gramática simplificada generaremos una
    función constructora.

    \begin{itemize}
      \item Prog $\rightarrow$ LDec \underline{SPROG} LIns\\
        prog: LDec x LIns $\rightarrow$ Prog
      \item LDec $\rightarrow$ LDec \underline{EOL} \underline{NUM}
        \underline{ID}\\
        ldec: LDec x String x String $\rightarrow$ LDec
      \item LDec $\rightarrow$ LDec \underline{EOL} \underline{BOOL}
        \underline{ID}\\
        ldec: LDec x String x String $\rightarrow$ LDec
      \item LDec $\rightarrow$ \underline{NUM} \underline{ID}\\
        dec: String x String $\rightarrow$ LDec
      \item LDec $\rightarrow$ \underline{BOOL} \underline{ID}\\
        dec: String x String $\rightarrow$ LDec
      \item LIns $\rightarrow$ LIns \underline{EOL} \underline{ID}
        \underline{IS} Exp\\
        lins: LIns x String x Exp $\rightarrow$ LIns
      \item LIns $\rightarrow$ \underline{ID} \underline{IS} Exp\\
        ins: String x Exp $\rightarrow$ LIns
      \item Exp $\rightarrow$ Exp \underline{PLUS} Exp\\
        plus: Exp x Exp $\rightarrow$ Exp
      \item Exp $\rightarrow$ Exp \underline{MINUS} Exp\\
        minus: Exp x Exp $\rightarrow$ Exp
      \item Exp $\rightarrow$ Exp \underline{AND} Exp\\
        and: Exp x Exp $\rightarrow$ Exp
      \item Exp $\rightarrow$ Exp \underline{OR} Exp\\
        or: Exp x Exp $\rightarrow$ Exp
      \item Exp $\rightarrow$ Exp \underline{EQ} Exp\\
        eq: Exp x Exp $\rightarrow$ Exp
      \item Exp $\rightarrow$ Exp \underline{GT} Exp\\
        gt: Exp x Exp $\rightarrow$ Exp
      \item Exp $\rightarrow$ Exp \underline{GEQ} Exp\\
        geq: Exp x Exp $\rightarrow$ Exp
      \item Exp $\rightarrow$ Exp \underline{LT} Exp\\
        lt: Exp x Exp $\rightarrow$ Exp
      \item Exp $\rightarrow$ Exp \underline{LEQ} Exp\\
        leq: Exp x Exp $\rightarrow$ Exp
      \item Exp $\rightarrow$ Exp \underline{NEQ} Exp\\
        neq: Exp x Exp $\rightarrow$ Exp
      \item Exp $\rightarrow$ Exp \underline{MUL} Exp\\
        mul: Exp x Exp $\rightarrow$ Exp
      \item Exp $\rightarrow$ Exp \underline{DIV} Exp\\
        div: Exp x Exp $\rightarrow$ Exp
      \item Exp $\rightarrow$ \underline{MINUS} Exp\\
        sign: Exp $\rightarrow$ Exp
      \item Exp $\rightarrow$ \underline{NOT} Exp\\
        not: Exp $\rightarrow$ Exp
      \item Exp $\rightarrow$ \underline{LREAL}\\
        num: String $\rightarrow$ Exp
      \item Exp $\rightarrow$ \underline{TRUE}\\
        true: String $\rightarrow$ Exp
      \item Exp $\rightarrow$ \underline{FALSE}\\
        false: String $\rightarrow$ Exp
    \end{itemize}

  \section{Diagrama de clases}
    \begin{tikzpicture}
      \node[rectangle, draw=black] (prog) {Prog};

      \node[rectangle, draw=black, above right=3 and .5of prog] (ldecs) {LDecs};

      \node[rectangle, draw=black, above right=2 and .5 of ldecs] (ldec) {ldec};

      \node[rectangle, draw=black, below right=2 and .5 of ldecs] (dec) {dec};

      \node[rectangle, draw=black, below right=3 and .5 of prog] (Lins) {LIns};

      \node[rectangle, draw=black, above right=2 and .5 of Lins] (lins) {lins};

      \node[rectangle, draw=black, below right=2 and .5 of Lins] (ins) {ins};

      \node[rectangle, draw=black, right=6 of Lins] (exp) {Exp};

      \node[rectangle, draw=black, above right=5 and 2 of exp] (expb) {ExpBin};

      \node[rectangle, draw=black, below right=5 and 3 of expb] (neq) {NEQ};
      \node[rectangle, draw=black, above=.5 of neq] (leq) {LEQ};
      \node[rectangle, draw=black, above=.5 of leq] (lt) {LT};
      \node[rectangle, draw=black, above=.5 of lt] (geq) {GEQ};
      \node[rectangle, draw=black, above=.5 of geq] (gt) {GT};
      \node[rectangle, draw=black, above=.5 of gt] (eq) {EQ};
      \node[rectangle, draw=black, above=.5 of eq] (and) {AND};
      \node[rectangle, draw=black, above=.5 of and] (or) {OR};
      \node[rectangle, draw=black, above=.5 of or] (plus) {PLUS};
      \node[rectangle, draw=black, above=.5 of plus] (minus) {MINUS};
      \node[rectangle, draw=black, above=.5 of minus] (mul) {MUL};
      \node[rectangle, draw=black, above=.5 of mul] (div) {DIV};

      \node[rectangle, draw=black, below right=3 and 2 of exp] (expu) {ExpUn};

      \node[rectangle, draw=black, above right=.5 and 1 of expu] (sign) {SIGN};
      \node[rectangle, draw=black, below right=.5 and 1 of expu] (not) {NOT};

      \node[rectangle, draw=black, below =6 of exp] (true) {TRUE};
      \node[rectangle, draw=black, left =.5 of true] (id) {ID};
      \node[rectangle, draw=black, left =.5 of id] (lreal) {LREAL};
      \node[rectangle, draw=black, right =.5 of true] (false) {FALSE};


      \path[->]
        (prog)
          edge (ldecs)
          edge (Lins)
        (lins) edge node[fill=white, anchor=center] {+} (exp)
        (ins) edge (exp)
        (expb.west) edge node[fill=white, anchor=center] {2} (exp)
        (expu.west) edge (exp)
        (dec) edge (exp)
        (ldec) edge (exp);

      \path[>= open triangle 60, ->]
        (ldec) edge (ldecs)
        (dec) edge (ldecs)
        (lins) edge (Lins)
        (ins) edge (Lins)
        (expb) edge (exp.east)
        (expu) edge (exp.east)
        (sign) edge (expu)
        (not) edge (expu)
        (true) edge (exp)
        (false) edge (exp)
        (id) edge (exp)
        (lreal) edge (exp)
        (div) edge (expb)
        (mul) edge (expb)
        (minus) edge (expb)
        (plus) edge (expb)
        (or) edge (expb)
        (and) edge (expb)
        (eq) edge (expb)
        (gt) edge (expb)
        (geq) edge (expb)
        (lt) edge (expb)
        (leq) edge (expb)
        (neq) edge (expb);
    \end{tikzpicture}

  \section{Gramática de atributos}
    S $\rightarrow$ Prog \underline{EOF}\\
    \hspace{5mm}S.a = Prog.a\\
    Prog $\rightarrow$ LDec \underline{SPROG} LIns\\
    \hspace{5mm}Prog.a = prog(LDec.a, LIns.a)\\
    LDec $\rightarrow$ LDec \underline{EOL} Dec\\
    \hspace{5mm}$LDec_0$.a = ldec($LDec_1$.a, Dec.ty, Dec.id)\\
    LDec $\rightarrow$ Dec\\
    \hspace{5mm}LDec.a = dec(Dec.ty, Dec.id)\\
    LIns $\rightarrow$ LIns \underline{EOL} Ins\\
    \hspace{5mm}$LIns_0$.a = lins($LIns_1$.a, Ins.id, Ins.exp)\\
    LIns $\rightarrow$ Ins\\
    \hspace{5mm}LIns.a = ins(Ins.id, Ins.exp)\\
    Dec $\rightarrow$ \underline{NUM} \underline{ID}\\
    \hspace{5mm}Dec.ty = NUM.lex\\
    \hspace{5mm}Dec.id = ID.lex\\
    Dec $\rightarrow$ \underline{BOOL} \underline{ID}\\
    \hspace{5mm}Dec.ty = BOOL.lex\\
    \hspace{5mm}Dec.id = ID.lex\\
    Ins $\rightarrow$ \underline{ID} \underline{IS} Exp0\\
    \hspace{5mm}Ins.id = ID.lex\\
    \hspace{5mm}Ins.exp = Exp0.a\\
    Exp0 $\rightarrow$ Exp0 Op0 Exp1\\
    \hspace*{5mm}$Exp0_0$.a = mkexpb(Op0.op, $Exp0_1$.a, Exp1.a)\\
    Exp0 $\rightarrow$ Exp1\\
    \hspace*{5mm}Exp0.a = Exp1.a\\
    Exp1 $\rightarrow$ Exp2 \underline{AND} Exp1\\
    \hspace*{5mm}$Exp1_0$.a = mkexpb('and', Exp2.a, $Exp1_1$.a)\\
    Exp1 $\rightarrow$ Exp2 \underline{OR} Exp2\\
    \hspace*{5mm}Exp1.a = mkexpb('or', $Exp2_0$.a, $Exp2_1$.a)\\
    Exp1 $\rightarrow$ Exp2\\
    \hspace*{5mm}Exp1.a = Exp2.a\\
    Exp2 $\rightarrow$ Exp3 Op2 Exp3\\
    \hspace*{5mm}Exp2.a = mkexpb(Op2.op, $Exp3_0$.a, $Exp3_1$.a)\\
    Exp2 $\rightarrow$ Exp3\\
    \hspace*{5mm}Exp2.a = Exp3.a\\
    Exp3 $\rightarrow$ Exp3 Op3 Exp4\\
    \hspace*{5mm}$Exp3_0$.a = mkexpb(Op3.a, $Exp3_1$.a, Exp4.a)\\
    Exp3 $\rightarrow$ Exp4\\
    \hspace*{5mm}Exp3.a = Exp4.a\\
    Exp4 $\rightarrow$ \underline{MINUS} Exp4\\
    \hspace*{5mm}$Exp4_0$.a = mkexpu('-', $Exp4_1$.a)\\
    Exp4 $\rightarrow$ \underline{NOT} Exp5\\
    \hspace*{5mm}Exp4.a = mkexpu('not', Exp5.a)\\
    Exp4 $\rightarrow$ Exp5\\
    \hspace*{5mm}Exp4.a = Exp5.a\\
    Exp5 $\rightarrow$ \underline{LREAL}\\
    \hspace*{5mm}Exp5.a = num(LREAL.lex)\\
    Exp5 $\rightarrow$ \underline{TRUE}\\
    \hspace*{5mm}Exp5.a = true(TRUE.lex)\\
    Exp5 $\rightarrow$ \underline{FALSE}\\
    \hspace*{5mm}Exp5.a = false(FALSE.lex)\\
    Exp5 $\rightarrow$ \underline{POP} Exp0 \underline{PCL}\\
    \hspace*{5mm}Exp5.a = Exp0.a\\
    Op0 $\rightarrow$ \underline{PLUS}\\
    \hspace*{5mm}Op0.op = '+'\\
    Op0 $\rightarrow$ \underline{MINUS}\\
    \hspace*{5mm}Op0.op = '-'\\
    Op2 $\rightarrow$ \underline{EQ}\\
    \hspace*{5mm}Op2.op = '=='\\
    Op2 $\rightarrow$ \underline{GT}\\
    \hspace*{5mm}Op2.op = '$>$'\\
    Op2 $\rightarrow$ \underline{GEQ}\\
    \hspace*{5mm}Op2.op = '$>$='\\
    Op2 $\rightarrow$ \underline{LT}\\
    \hspace*{5mm}Op2.op = '$<$'\\
    Op2 $\rightarrow$ \underline{LEQ}\\
    \hspace*{5mm}Op2.op = '$<$='\\
    Op2 $\rightarrow$ \underline{NEQ}\\
    \hspace*{5mm}Op2.op = '!='\\
    Op3 $\rightarrow$ \underline{MUL}\\
    \hspace*{5mm}Op3.op = '*'\\
    Op3 $\rightarrow$ \underline{DIV}\\
    \hspace*{5mm}Op3.op = '/'\\

    \bigskip
    \par
    Las funciones mkexpb y mkexpu son las encargadas de construir las expresiones
    apropiadas para operadores binarios y unarios, respectivamente.\\
    \vspace{2mm}
    \textbf{fun} mkexpb (op, opd1, opd2) \{\\
    \hspace{5mm}switch(op) \{\\
    \hspace{10mm}'+': return plus(opd1, opd2)\\
    \hspace{10mm}'-': return minus(opd1, opd2)\\
    \hspace{10mm}'*': return mul(opd1, opd2)\\
    \hspace{10mm}'/': return div(opd1, opd2)\\
    \hspace{10mm}'and': return and(opd1, opd2)\\
    \hspace{10mm}'or': return or(opd1, opd2)\\
    \hspace{10mm}'==': return eq(opd1, opd2)\\
    \hspace{10mm}'$>$': return gt(opd1, opd2)\\
    \hspace{10mm}'$>$=': return geq(opd1, opd2)\\
    \hspace{10mm}'$<$': return lt(opd1, opd2)\\
    \hspace{10mm}'$<$=': return leq(opd1, opd2)\\
    \hspace{10mm}'!=': return neq(opd1, opd2)\\
    \hspace{5mm}\}\\
    \}\\
    \vspace{2mm}
    \textbf{fun} mkexpu (op, opd) \{\\
    \hspace{5mm}switch(op) \{\\
    \hspace{10mm}'-': return sign(opd)\\
    \hspace{10mm}'not': return not(opd)\\
    \hspace{5mm}\}\\
    \}\\

  \section{Acondicionamiento para implementación descendente}
    S $\rightarrow$ Prog \underline{EOF}\\
    \hspace{5mm}S.a = Prog.a\\
    Prog $\rightarrow$ SDec \underline{SPROG} SIns\\
    \hspace{5mm}Prog.a = prog(LDec.a, LIns.a)\\
    SDec $\rightarrow$ Dec LDec\\
    \hspace{5mm}LDec.ah = dec(Dec.ty, Dec.id)\\
    \hspace{5mm}SDec.a = LDec.a\\
    LDec $\rightarrow$ \underline{EOL} Dec LDec\\
    \hspace{5mm}$LDec_1$.ah = ldec($LDec_0$.ah, Dec.ty, Dec.id)\\
    \hspace{5mm}$LDec_0$.a = $LDec_1$.a\\
    LDec $\rightarrow$ $\varepsilon$\\
    \hspace{5mm}LDec.a = LDec.ah\\
    SIns $\rightarrow$ Ins LIns\\
    \hspace{5mm}LIns.ah = ins(Ins.id, Ins.exp)\\
    \hspace{5mm}SIns.a = LIns.a\\
    LIns $\rightarrow$ \underline{EOL} Ins LIns\\
    \hspace{5mm}$LIns_1$.ah = lins($LIns_0$.ah, Ins.id, Ins.exp)\\
    \hspace{5mm}$LIns_0$.a = $LIns_1$.a\\
    LIns $\rightarrow$ $\varepsilon$\\
    \hspace{5mm}LIns.a = LIns.ah\\
    Dec $\rightarrow$ \underline{NUM} \underline{ID}\\
    \hspace{5mm}Dec.ty = NUM.lex\\
    \hspace{5mm}Dec.id = ID.lex\\
    Dec $\rightarrow$ \underline{BOOL} \underline{ID}\\
    \hspace{5mm}Dec.ty = BOOL.lex\\
    \hspace{5mm}Dec.id = ID.lex\\
    Ins $\rightarrow$ \underline{ID} \underline{IS} Exp0\\
    \hspace{5mm}Ins.id = ID.lex\\
    \hspace{5mm}Ins.exp = Exp0.a\\
    Exp0 $\rightarrow$ Exp1 R0\\
    \hspace{5mm}R0.ah = Exp1.a\\
    \hspace{5mm}Exp0.a = R0.a\\
    R0 $\rightarrow$ Op0 Exp1 R0\\
    \hspace{5mm}$R0_1$.ah = mkexpb(Op0.op, $R0_0$.ah, Exp1.a)\\
    \hspace{5mm}$R0_0$.a = $R0_1$.a\\
    R0 $\rightarrow$ $\varepsilon$\\
    \hspace{5mm}R0.a = R0.ah\\
    Exp1 $\rightarrow$ Exp2 R1\\
    \hspace{5mm}R1.ah = Exp2.a\\
    \hspace{5mm}Exp1.a = R1.a\\
    R1 $\rightarrow$ \underline{AND} Exp2 R1\\
    \hspace{5mm}$R1_1$.ah = mkexpb('and', $R1_0$.ah, Exp2.a)\\
    \hspace{5mm}$R1_0$.a = $R1_1$.a\\
    R1 $\rightarrow$ \underline{OR} Exp2\\
    \hspace{5mm}R1.a = mkexpb('or', R1.ah, Exp2.a)\\
    R1 $\rightarrow$ $\varepsilon$\\
    \hspace{5mm}R1.a = R1.ah\\
    Exp2 $\rightarrow$ Exp3 R2\\
    \hspace{5mm}R2.ah = Exp3.a\\
    \hspace{5mm}Exp2.a = R2.a\\
    R2 $\rightarrow$ Op2 Exp3 R2\\
    \hspace{5mm}$R2_1$.ah = mkexpb(Op2.op, $R2_0$.ah, Exp3.a)\\
    \hspace{5mm}$R2_0$.a = $R2_1$.a\\
    R2 $\rightarrow$ $\varepsilon$\\
    \hspace{5mm}R2.a = R2.ah\\
    Exp3 $\rightarrow$ Exp4 R3\\
    \hspace{5mm}R3.ah = Exp4.a\\
    \hspace{5mm}Exp3.a = R3.a\\
    R3 $\rightarrow$ Op3 Exp4 R3\\
    \hspace{5mm}$R3_1$.ah = mkexpb(Op3.op, $R3_0$.ah, Exp4.a)\\
    \hspace{5mm}$R3_0$.a = $R3_1$.a\\
    R3 $\rightarrow$ $\varepsilon$\\
    \hspace{5mm}R3.a = R3.ah\\
    Exp4 $\rightarrow$ \underline{MINUS} Exp4\\
    \hspace{5mm}$Exp4_0$.a = mkexpu('-', $Exp4_1$.a)\\
    Exp4 $\rightarrow$ \underline{NOT} Exp5\\
    \hspace{5mm}Exp4.a = mkexpu('not', Exp5.a)\\
    Exp4 $\rightarrow$ Exp5\\
    \hspace{5mm}Exp4.a = Exp5.a\\
    Exp5 $\rightarrow$ \underline{LREAL}\\
    \hspace*{5mm}Exp5.a = num(LREAL.lex)\\
    Exp5 $\rightarrow$ \underline{TRUE}\\
    \hspace*{5mm}Exp5.a = true(TRUE.lex)\\
    Exp5 $\rightarrow$ \underline{FALSE}\\
    \hspace*{5mm}Exp5.a = false(FALSE.lex)\\
    Exp5 $\rightarrow$ \underline{POP} Exp0 \underline{PCL}\\
    \hspace*{5mm}Exp5.a = Exp0.a\\
    Op0 $\rightarrow$ \underline{PLUS}\\
    \hspace*{5mm}Op0.op = '+'\\
    Op0 $\rightarrow$ \underline{MINUS}\\
    \hspace*{5mm}Op0.op = '-'\\
    Op2 $\rightarrow$ \underline{EQ}\\
    \hspace*{5mm}Op2.op = '=='\\
    Op2 $\rightarrow$ \underline{GT}\\
    \hspace*{5mm}Op2.op = '$>$'\\
    Op2 $\rightarrow$ \underline{GEQ}\\
    \hspace*{5mm}Op2.op = '$>$='\\
    Op2 $\rightarrow$ \underline{LT}\\
    \hspace*{5mm}Op2.op = '$<$'\\
    Op2 $\rightarrow$ \underline{LEQ}\\
    \hspace*{5mm}Op2.op = '$<$='\\
    Op2 $\rightarrow$ \underline{NEQ}\\
    \hspace*{5mm}Op2.op = '!='\\
    Op3 $\rightarrow$ \underline{MUL}\\
    \hspace*{5mm}Op3.op = '*'\\
    Op3 $\rightarrow$ \underline{DIV}\\
    \hspace*{5mm}Op3.op = '/'\\
\end{document}
