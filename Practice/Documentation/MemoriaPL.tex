\documentclass[11pt]{article}
\usepackage[utf8]{inputenc}

%%% PAGE DIMENSIONS
\usepackage{geometry}
\geometry{a4paper}

%%% PACKAGES
\usepackage{booktabs}
\usepackage{paralist}
\usepackage{verbatim}
\usepackage{subfig}
\usepackage{graphicx}
\usepackage{nameref}
\usepackage[colorlinks = true,
            linkcolor = black,
            urlcolor  = blue,
            citecolor = blue,
            anchorcolor = blue]{hyperref}
\usepackage{subfiles}
\usepackage{color}
\usepackage{placeins}
\usepackage{nameref}
\usepackage{chngcntr}
\usepackage{tikz}
\usepackage[colorlinks = true,
            linkcolor = black,
            urlcolor  = blue,
            citecolor = blue,
            anchorcolor = blue]{hyperref}
\usepackage[spanish]{cleveref}
\usepackage{amssymb}

\usetikzlibrary{positioning,shapes,arrows,chains,automata}

\providecommand{\main}{.}

\counterwithin{figure}{section}
\counterwithin{table}{section}

%%% SECTION TITLE APPEARANCE
\usepackage{sectsty}
\allsectionsfont{\sffamily\mdseries\upshape}

%%% ToC (table of contents) APPEARANCE
\renewcommand{\contentsname}{Contenidos}
\usepackage[nottoc,notlof,notlot]{tocbibind}
\usepackage[titles,subfigure]{tocloft}
\renewcommand{\cftsecfont}{\rmfamily\mdseries\upshape}
\renewcommand{\cftsecpagefont}{\rmfamily\mdseries\upshape}

\counterwithin*{figure}{section}
\counterwithin*{figure}{subsection}
\counterwithin*{figure}{subsubsection}

\counterwithin*{table}{section}
\counterwithin*{table}{subsection}
\counterwithin*{table}{subsubsection}

\addtolength{\cftfignumwidth}{2em}

\renewcommand{\thefigure}{
  \ifnum\value{subsection}=0
    \thesection.\arabic{figure}
  \else
    \ifnum\value{subsubsection}=0
      \thesubsection.\arabic{figure}
    \else
      \thesubsubsection.\arabic{figure}
    \fi
  \fi
}

\renewcommand{\thetable}{
  \ifnum\value{subsection}=0
    \thesection.\arabic{table}
  \else
    \ifnum\value{subsubsection}=0
      \thesubsection.\arabic{table}
    \else
      \thesubsubsection.\arabic{table}
    \fi
  \fi
}

%%% END Article customizations

%%% The "real" document content comes below...

\title{Práctica\\\Large Procesadores de Lenguajes}
\author{David Antuña Rodríguez\\Javier Carrión García}
\date{}

\begin{document}
  \tikzstyle{sheet}= [rectangle, draw = black, minimum width = .8cm, minimum height = 1.2cm]
  \tikzstyle{unit}= [rectangle, draw = blue, minimum width = 2.4cm, minimum height = 1.6cm]
  \tikzstyle{clo} = [cloud, draw, cloud puffs = 10, cloud puff arc = 120, aspect = 1.5, inner ysep=1em]
  \tikzstyle{err2} = [regular polygon, regular polygon sides=3, draw = red, fill=white, text width=1em, inner sep=1mm, outer sep=0mm, shape border rotate=0 ]
  \tikzstyle{txt}= [rectangle, draw = white, align = center]
  \tikzstyle{err}= [rectangle, draw = red, minimum width = .8cm, minimum height = 1.2cm]

  \tikzstyle{myarrow}=[->, >=open triangle 90, thick]
  \tikzstyle{myarrow2}=[->, >=latex', thick,dashed]

  \raggedright
  % Title and Table of Contents
  \pagenumbering{gobble}
  \maketitle
  \newpage
  \tableofcontents
  \newpage
  \pagenumbering{arabic}

  \section{Fase 1}
  {
    \let\section\subsection
    \let\subsection\subsubsection
    \subfile{Lexer/Lexer}
  }

  \section{Fase 2}
  {
    \let\section\subsection
    \let\subsection\subsubsection
    \subfile{Parser/Parser}
  }

  \section{Fase 3}
  {
    \par
    En esta fase solo es necesaria la implementación del analizador sintáctico
    LR con JLex + CUP.
  }

  \section{Fase 4}
  {
    \let\section\subsection
    \let\subsection\subsubsection
    \subfile{Abstract_Syntax/Abstract_Syntax}
  }
\end{document}